% 4 Resultater
%Her skal resultatene beskrives i en nøktern stil. Diskusjoner og vurderinger kommer i neste kapittel. Du finner eksempel og nærmere beskrivelse av dette i boka til Rognsaa, Aage. Bacheloroppgaven : Skriveråd Og Regler for Utformingen. Oslo: Universitetsforl, 2015.
%Dette kapitlet deles i (minst) tre deler:
%Vitenskapelige resultater: Beskrive data/empiri/produkt/design som blir underlag til svar på problemstilling og eventuelle hypoteser fra kapittel 1. Resultatene skal vise et systematisk og etterrettelig arbeid. 
%Ingeniørfaglige resultater: Ta for deg målene som ble satt i begynnelsen av prosjektet. Målene i et systemutviklingsprosjekt vil være beskrevet i visjondokumentet/forprosjektet i vedlegg. Beskriv status for hvert av disse målene. Som del av dette vil det være naturlig å beskrive status på systemet ved leveringstidspunktet.
%Resultater fra ulike typer tester hører hjemme her. Eventuelt med detaljer i vedlegg.
%Administrative resultater: Her er prosjektdagboka (legges ved som vedlegg) et nyttig verktøy.
% Skriv om måloppfyllelse i forhold til framdriftsplan: Planen, slik den var tidlig i prosjektet, og virkeligheten. (Eventuelle kommentarer og forklaringer på at ting ble som de ble, skal skrives i neste kapittel.)
%Timeregnskap, samlet fordelt på timeforbruk og aktiviteter. Referer til kalendertid dersom det er relevant.
%Studenter med systemutviklingsprosjekt må dokumentere at utviklingsprosessen de har valgt virkelig er brukt, ved å beskrive hva som har skjedd når og til hvilke tidspunkter. Detaljene legges i vedlegg. For tyngre prosesser skal de enkelte trinnene framgå av timelister og statusrapporter.