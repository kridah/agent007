%5 Diskusjon

% Det er naturlig å dele dette kapitlet på samme måte som kapittel 4. Her skal dere drøfte årsaker til at resultatene ble som de ble, spesielt der det er avvik fra planer og oppsatte mål.

% Hvorfor holdt hypotesene, eller hvorfor holdt ikke hypotesene? Drøft hvordan resultatene kan forstås i forhold til eller som svar på problemstillingen. Hvordan ble sluttproduktet? Fikk oppdragsgiver det som var forventet? Hvilke krav ble oppfylt? Hvilke krav ble ikke oppfylt? Hvorfor ble resultatene som de ble? Hva var bra? Hva var ikke så bra? Hva ble bra på grunn av valgt prosess, fremgangsmåte og teknologi? Hva ble ikke bra på grunn av valgt prosess, fremgangsmåte og teknologi? Hva ble bra eller dårlig uavhengig av valgt prosess, fremgangsmåte og teknologi?

% ”I tillegg bør du her peke på både svakheter og styrker ved oppgaveløsningen din. Tro ikke at sensor ikke kommer til å se eventuelle svakheter – vær heller føre vár og kommentér dette selv. Hvis du viser at du er klar over svakhetene ved ditt eget arbeid, men kan forklare disse og gi anbefalinger til videre arbeid med liknende oppgaver, vil du kunne snu dette til en styrke.”[1]

% Dere skal også drøfte arbeidet i forhold til et helhetlig systemperspektiv. Sett resultatene inn i en samfunnsmessig og økonomisk, eventuelt også miljømessig, sammenheng. Analyser relevante etiske problemstillinger i forhold til resultatene fra arbeidet. Kompendiet under referanse [2] kan være nyttig. I diskusjonen bør også "kommersialiteten" i produktet vurderes.

\chapter{Diskusjon}
