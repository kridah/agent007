%6 Konklusjon og videre arbeid

% Her presenteres de konklusjoner som kan trekkes i forhold til stilte hypoteser/problemstillinger (kapittel 1) samt kravene i visjonsdokumentet, gitt diskusjonen i forrige kapittel. Hvis arbeidet har endret seg underveis, slik at dere ikke kan svare på den opprinnelige formulerte hypotese/problemstilling, er det mulig å endre disse. (Husk at et negativt svar også kan være et riktig svar på en hypotese.)

% ”Konklusjonene skal ikke være bare ’den siste delen i rapporten’. Konklusjonen skal ikke være en oppsummering. Den skal være gyldige ytringer og forklaringer som følger direkte av resultatene og diskusjonen. Du kan også gjerne inkludere anbefalinger for personer som skal gjøre liknende oppgaver senere, eller bygge på arbeidet du har gjort.” [1]

\chapter{Konklusjon}
