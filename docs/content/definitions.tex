\definition{User story} Sier noe om hvordan brukeren ønsker at programvaren skal fungere.
\definition{UML} Unified Modeling Language (UML) er en industristandard for datarelatert modellering.
\definition{Prototype} En prototyp eller prototype (av gresk, «førsteinntrykk») er en foreløpig utgave av et produkt.
\definition{Klassediagram} Et klassediagram viser en statisk oversikt over koden. Den viser hvordan klassene er relatert og om de bruker hverandre. Et objekt diagram viser en dynamisk oversikt over selve objektet.
\definition{Kardiolog} En kardiolog defineres som en lege med spesialisering på hjertet og sykdommer relatert til hjertet. I dagligtalen brukes ofte hjertespesialist.
\definition{PBL} Product backlog er en produktkø av oppgaver et utviklingsteam skal gjennomføre i en sprint
\definition{Systemadministrator} En systemadministrator har større rettigheter enn vanlige brukere
\definition{Konsultasjon} Konsultasjon er et møte hvor man søker råd for noe, som for eksempel hos lege
\definition{Skytjenester} Skytjenester er en samlebetegnelse på alt fra dataprosessering og datalagring til programvare på servere som er tilgjengelig fra eksterne serverparker tilknyttet internett.
\definition{Brukergrensesnitt} er en betegnelse på kontaktflaten mellom brukeren og datamaskinens operativsystem og programmer, som avgjør hvordan brukeren styrer programmet
\definition{Container} "Container" er et diskret miljø satt opp i et operativsystem der ett eller flere applikasjoner kan kjøres, tildeles vanligvis bare de ressursene som er nødvendige for at applikasjonen skal fungere riktig.
\definition{Entitet} En entitet er noe selvstendig og entydig – som har eller kan gis en entydig identifikator
\definition{Backend} Backend er den delen av programvaren som ligger nærmest databasen der dataene er lagret.
\definition{Algoritme} er en fullstendig og nøyaktig beskrivelse av fremgangsmåten for løsning av en beregningsoppgave eller annen oppgave.
\definition{Metadata} Data som beskriver data




 





