Introduksjon
Denne malen er blant annet basert på tilsvarende maler utviklet ved NTNU og gjengitt etter tillatelse fra forfatterne (E. Lervik m.fl.).

Hovedrapporten skal dokumentere bacheloroppgaven. Vær obs på måten dere skriver på. Dere skal skrive en faglig rapport og ikke en «Fortellingen om hvordan vi gjennomførte bacheloroppgaven».

Et vanlig spørsmål fra studentene er hvor mye tilbakemeldinger de kan forvente å få underveis på dokumentasjon og rapporter de leverer inn. Tilbakemeldinger gis normalt via formelle møter som skal gjennomføres i løpet av prosjektet.

Ellers gjelder følgende for systemutviklingsprosjekter (som er de aller fleste):

Veileder gir tilbakemelding på forprosjektet én gang.
Det vil være naturlig å levere inn kravdokumentasjonen i flere omganger. Da er hovedregelen at veileder gir tilbakemelding på nye kapitler.
Veileder gir, som hovedregel, EN tilbakemelding på hovedrapporten, men skal selvfølgelig for øvrig kunne gi tips og råd underveis.
Unntak avtales med veileder.

Med hensyn til faglig nivå på de som skal lese de dokumentene dere produserer, så kan dere, dersom oppgavestiller ikke krever noe annet, gå ut fra at dette er studenter som har grunnleggende kunnskaper i informatikk. Unntaket er eventuell brukerdokumentasjon som må tilpasses den eller de brukergruppene som systemet lages for.

Merk at rekkefølgen av forfatternavnene på et dokument forteller noe. Dersom rekkefølgen er alfabetisk på etternavn, betyr det at alle har bidratt likt. Dersom rekkefølgen ikke er alfabetisk, vil leseren anta at forfatterne er satt opp i rekkefølge etter bidrag. For hovedrapporten fra en bacheloroppgave er det vanligvis slik at alle har bidratt likt – om ikke akkurat til setningsformuleringene, så til oppgaven som helhet. Dersom det ikke er slik, skal forklaring framkomme av refleksjonsnotater eller på annen måte. For de øvrige dokumentene (vedleggene til hovedrapporten) kan det forsvares at rekkefølgen varierer fra dokument til dokument.

Generelt:

Rapporten uten vedlegg bør være på 20-50 sider.
Nedenfor er rapportens disposisjon på øverste kapittelnivå gitt.Velg selv å dele inn i passende underkapitler.
Som rapportforside skal standard for bacheloroppgaver benyttes (ligger på LMS/Canvas).
Overskriftene i malen under er identiske med kapitteloverskriftene dere bør bruke i rapporten.

Foretrukket mal
Følgende mal er foretrukket på avsluttende rapport. Du finner et ferdig formatert oppstarts Word-dokument i dokumentet Bacheloroppgave MAL IDBI 2020-1.docxForhåndsvis dokumentet Bruk gjerne dette som utgangspunkt.

 

Forside

Husk å fyllet ut "tabellen" på siden etter forsiden

Forord

Hvorfor ble oppgaven valgt? Skriv kort om prosessen som har ført fram til resultatet. Husk å takke for hjelp og støtte fra ulike hold.

Dato, sted, navn og underskrift av alle prosjektdeltakerne.

Et forord i en rapport av denne typen bør ikke være på mer enn én side.

Innholdsfortegnelse

Dette skal inn mellom Sammendrag og Kapittel 1. Husk at også vedleggene med sidetall skal inn i innholdsfortegnelsen. Enten sidenummererer dere alt fortløpende, eller så lar dere vedleggene få sidenummer som følger: A-1, A-2, … B-1, B-2. Sørg for at det er lett for sensor å finne fram i papirutgaven av rapporten, inkludert vedleggene!

Sammendrag

Dette er et mer omfattende sammendrag enn det som er på forsiden av rapporten. Ca 1 side. Sammendraget er viktig da det ofte er avgjørende for om resten av rapporten blir lest.

1 Introduksjon og relevans

De fleste oppgavene våre er blitt til fordi en oppdragsgiver har et behov som skal dekkes. Oppdragsgiver har ønsker om at studenter skal utvikle et produkt. Det er likevel en grunn til dette behovet. Gjør kort rede for dette her. Relater gjerne til om dette f.eks. er effektmål, prosessmål eller produktmål. Det bør også være et pkt. om avgrensninger i oppgaven

I denne delen av rapporten skal dere også sette opp problemstilling og eventuelle hypoteser som dere skal utforske.

Første utgave av dette lages i forprosjekt/visjonsdokumentet det obligatoriske arbeidskravet tidlig i semesteret.

Husk også å beskrive rapportens struktur. Det vil si en oversikt over hva de forskjellige kapitlene inneholder.

2 Teori: drøfting av mulige teknologier og utviklingsmetoder

Eventuelt litteraturstudium skal inn her. Her skal dere beskrive og forklare den teoretiske bakgrunnen for arbeidet dere gjør (ikke nødvendigvis bare for problemstillingen i kapittel 1).  Hva dere skal skrive om vil avhenge av oppgaven dere skal løse. Noen eksempler på overskrifter fra tidligere oppgaver: strekkoder, autentisering, køteori, feilmedisinering. Det siste er et eksempel på såkalt domenekunnskap, det vil si teori knyttet til bruken av resultatene. Ta med referanser til lærebok eller annen litteratur der det er relevant.

Produktnavn hører ikke hjemme i dette kapitlet. Eksempel: Dere kan skrive om relasjonsdatabaser, men ikke om Oracle og MySQL.

Vær ryddig i disposisjonen av kapitlet. Det er spesielt viktig dersom deler av kapitlet ikke er direkte knyttet til problemstillingen i kapittel 1.

3 Valg av teknologi og utviklingsmetode

I kapittel 2 beskrev dere mulige teknologier og utviklingsmetoder for å gjennomføre det ingeniørfaglige arbeidet (løse oppgaven). I dette kapitlet skal dere begrunne de valgene dere har gjort. Begrunnelsen skal relateres til utforskning av problemstillingen og svar på eventuelle hypoteser fra kapittel 1. For systemutviklingsprosjekter skal valg av teknologi og utviklingsprosess/utviklingsmetode (også) begrunnes i forhold til kravene satt opp i forprosjekt/visjonsdokumentet.

Ellers kan kapitlet omfatte algoritmer, overordnet organisering av klasser og referansearkitektur. Del opp kapitlet i underkapitler for hver type valg. Ta med referanser til lærebok eller annen litteratur der det er relevant.

Vær tilbakeholdende med omfattende produktbeskrivelser. Om dere velger å bruke et bestemt databasesystem framfor et annet kan dere imidlertid kort ta med hva som er årsaken til dette valget.

Om arbeidet utføres av flere studenter i fellesskap, skal det være med et underkapittel om arbeids- og rollefordeling.

Erfaring viser at enkelte studenter/grupper blir veldig personlige i dette kapitlet. Dere skal altså ikke skrive historien om prosjektgjennomføringen som begynte med oppstartmøtet og skal avsluttes med presentasjon. Også her skal dere være saklige og nøkterne. Husk igjen at dette er en faglig rapport og ikke norsk barneskolestil.

4 Resultater

Her skal resultatene beskrives i en nøktern stil. Diskusjoner og vurderinger kommer i neste kapittel. Du finner eksempel og nærmere beskrivelse av dette i boka til Rognsaa, Aage. Bacheloroppgaven : Skriveråd Og Regler for Utformingen. Oslo: Universitetsforl, 2015.

Dette kapitlet deles i (minst) tre deler:

Vitenskapelige resultater: Beskrive data/empiri/produkt/design som blir underlag til svar på problemstilling og eventuelle hypoteser fra kapittel 1. Resultatene skal vise et systematisk og etterrettelig arbeid. 
Ingeniørfaglige resultater: Ta for deg målene som ble satt i begynnelsen av prosjektet. Målene i et systemutviklingsprosjekt vil være beskrevet i visjondokumentet i vedlegg. Beskriv status for hvert av disse målene. Som del av dette vil det være naturlig å beskrive status på systemet ved leveringstidspunktet.
Resultater fra ulike typer tester hører hjemme her. Eventuelt med detaljer i vedlegg.
Administrative resultater: Her er prosjektdagboka (legges ved som vedlegg) et nyttig verktøy.

Skriv om måloppfyllelse i forhold til framdriftsplan: Planen, slik den var tidlig i prosjektet, og virkeligheten. (Eventuelle kommentarer og forklaringer på at ting ble som de ble, skal skrives i neste kapittel.)

Timeregnskap, samlet fordelt på timeforbruk og aktiviteter. Referer til kalendertid dersom det er relevant.

Studenter med systemutviklingsprosjekt må dokumentere at utviklingsprosessen de har valgt virkelig er brukt, ved å beskrive hva som har skjedd når og til hvilke tidspunkter. Detaljene legges i vedlegg. For tyngre prosesser skal de enkelte trinnene framgå av timelister og statusrapporter.

For smidige utviklingsprosesser brukes user stories, backlog, sprints o.l. som beskrevet i Mal for kravdokument.

5 Diskusjon

Det er naturlig å dele dette kapitlet på samme måte som kapittel 4. Her skal dere drøfte årsaker til at resultatene ble som de ble, spesielt der det er avvik fra planer og oppsatte mål.

Hvorfor holdt hypotesene, eller hvorfor holdt ikke hypotesene? Drøft hvordan resultatene kan forstås i forhold til eller som svar på problemstillingen. Hvordan ble luttproduktet? Fikk oppdragsgiver det som var forventet? Hvilke krav ble oppfylt? Hvilke krav ble ikke oppfylt? Hvorfor ble resultatene som de ble? Hva var bra? Hva var ikke så bra? Hva ble bra på grunn av valgt prosess, fremgangsmåte og teknologi? Hva ble ikke bra på grunn av valgt prosess, fremgangsmåte og teknologi? Hva ble bra eller dårlig uavhengig av valgt prosess, remgangsmåte og teknologi?

”I tillegg bør du her peke på både svakheter og styrker ved oppgaveløsningen din. Tro ikke at sensor ikke kommer til å se eventuelle svakheter – vær heller føre vár og kommentér dette selv. Hvis du viser at du er klar over svakhetene ved ditt eget arbeid, men kan forklare disse og gi anbefalinger til videre arbeid med liknende oppgaver, vil du kunne snu dette til en styrke.”[1]

Dere skal også drøfte arbeidet i forhold til et helhetlig systemperspektiv. Sett resultatene inn i en samfunnsmessig og økonomisk, eventuelt også miljømessig, sammenheng. Analyser relevante etiske problemstillinger i forhold til resultatene fra arbeidet. Kompendiet under referanse [2] kan være nyttig. I diskusjonen bør også "kommersialiteten" i produktet vurderes.

6 Konklusjon og videre arbeid

Her presenteres de konklusjoner som kan trekkes i forhold til stilte hypoteser/problemstillinger (kapittel 1) samt kravene i visjonsdokumentet, gitt diskusjonen i forrige kapittel. Hvis arbeidet har endret seg underveis, slik at dere ikke kan svare på den opprinnelige formulerte hypotese/problemstilling, er det mulig å endre disse. (Husk at et negativt svar også kan være et riktig svar på en hypotese.)

”Konklusjonene skal ikke være bare ’den siste delen i rapporten’. Konklusjonen skal ikke være en oppsummering. Den skal være gyldige ytringer og forklaringer som følger direkte av resultatene og diskusjonen. Du kan også gjerne inkludere anbefalinger for personer som skal gjøre liknende oppgaver senere, eller bygge på arbeidet du har gjort.” [1]

7 Referanser

Her setter dere opp all litteratur (inkl. egne rapporter fra høstprosjektet) dere henviser til i teksten. Se eget kapittel i begynnelsen av dette notatet.

Referanser er viktig, ta heller med for mange enn for få - men husk at det ikke skal være med referanser som det IKKE er refert til i rapporten. Referanser skal være med og vise sensor at dere har jobbet grundig med stoffet. Referanser til Wikipedia og andre publikasjoner på nett skal selvfølgelig også være med, hvis dere har brukt dem, men sensor lar seg neppe imponere av slike. Referanser med ISBN nummer er det som teller. Referanser skal ved første øyekast gi leseren innsikt i hva slags kilde det er – og skal om slikt finnes, ha et forfatternavn og årstall, samt publiseringssted. URL er sjelden tilstrekkelig. Videre skal det være en tydelig kobling, frem og tilbake, mellom aktuelt sted i rapporten og referanser.

8 Liste over figurer

9 Liste over tabeller

10 Liste over akronymer og forkortelser

11 Stikkordsliste

12 Vedlegg

Oppgavetekst: Erfaring viser at oppgavetekster endres underveis. Sett inn oppgaveteksten slik den var opprinnelig dersom det fortsatt er i samsvar med virkeligheten. Ellers gjør dere kort rede for oppgaven slik den opprinnelig var, og slik den har utviklet seg underveis. Hvorfor ting har blitt som de har blitt, hører hjemme andre steder i rapporten – det kommer litt an på hva det er.

Forprosjekt: Legg ved siste versjon

Kravdokumentasjon, evt. systemdokumentasjon, samt relevante deler av underveisrapporteringen, m. m.

Husk at dere i hovedrapporten skal henvise til hvert vedlegg minst en gang. Vedlegg det ikke henvises til skal ikke være med.